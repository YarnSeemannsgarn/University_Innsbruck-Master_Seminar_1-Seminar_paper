\section{Introduction}

The global Internet traffic is dominated by video streaming and downloading. In 2015  approximately 66\% (27 exabytes per month) of all consumer Internet traffic had to do with Internet videos, excluding \gls{p2p} video file downloads. Furthermore this number is expected to increase to 80\% (89 exabytes per month) until 2019 \cite{misc:cisco}. Hence the Internet video streaming \& downloading providers  need scaling architectures to handle this amount of traffic.
\\
\\
The traffic can be distinguished between delivery of live video streaming with on-the-fly encoding, like \gls{iptv} or Skype, and delivery of pre-encoded videos, so called \gls{vod}. The most famous \gls{vod} portal is YouTube, which already had more than 4 billion views a day in 2012 \cite{misc:scalibility_at_youtube}. More recent numbers are difficult to get, because YouTube itself does not reveal actual view numbers, but the YouTube press site states that every day people watch hundreds of millions of hours on YouTube and generate billions of views \cite{misc:youtube_press}.
\\
\\
This seminar work focuses on investigating both the hardware and the software aspects of YouTube's video delivery process, even if the underlying infrastructure and the used software components are only partly unveiled by its owner - Google Inc. Fortunately the head architect of YouTube Mike Solomon gave a speech at the \gls{pycon} in 2012 about the scalability of YouTube \cite{misc:scalibility_at_youtube} and most of the concepts mentioned there are still present. Moreover some papers about the streaming platform have been published in the last couple of years, most of them discussing performance measurements \cite{inpr:vivisecting_youtube, inp:network_characteristics} and \gls{qoe} \cite{inc:video_delivery, inc:transport_protocols}, but some of them also addressing the server infrastructure \cite{inpr:server_selection, inpr:youtube_everywhere}.
