\section{Summary and Conclusion}

In this seminar paper the delivery process of YouTube videos to client devices was observed. Therefore a closer look was taken both on the hardware and the software aspects of this process. \\
\\
In chapter \ref{chp:hardware_side} the delivery infrastructure - in other words the hardware side of the delivery process - was described. Most significant here is the usage of different server clusters for the transfer of the web site with the video container and the video content itself. The cache servers are organized in a three tier hierarchy to enable load balancing and faster delivery of popular videos. Despite this the redirections between the cache servers introduce additional traffic which must be considered in the delivery process.\\
\\
%TODO: check propreritar
Aspects of the software side were explained in chapter \ref{sec:software_side}. That included the software stack which is used or at least was used in 2012 by YouTube for their servers. Very notable is that they use widely spread components like the Apache HTTP Server and MySQL rather than proprietary problem-optimized software. This is probably due to their goal to keep things as simple as possible as long as they scale. Furthermore they use considerable scalability techniques like the correctness approximation and the addition of entropy to enable things like view counters and comments for billions of user actions per day. Last but not least the traffic management of a video transfer depends on different factors like the device and the browser which are used. These factors affect the buffering and the steady state phase of the transfer.\\
\\
On the hardware side it will be interesting to see, whether the server infrastructure with the front end web servers and the hierarchical cache servers will work properly for the increasing web traffic and amount of stored data. Google at least has the necessary means to add more and more servers to its infrastructure. On the software side the official current and future software stacks would be interesting to observe, because right now Google's only officially announced self-written and massively used software component for YouTube is Vitess, which is nevertheless also open source. By using wide-spread open source software a lot, YouTube is one of the best examples, that scalability rather depends on the right design and configuration than on the used software itself.
